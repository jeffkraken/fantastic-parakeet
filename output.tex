\documentclass{article}
\usepackage[utf8]{inputenc}
\usepackage{courier}
\usepackage{geometry}
\geometry{margin=1in}

\titlePython Best Practices Cheat Sheet
\authorJeff Krakenberg
\dateApril 18, 2025

\begin{document}

\maketitle


\section*1. Docstrings

Use triple quotes to explain what a function, class, or script does.

\begin{verbatim}
def scan_ports(ip):
    """Scan all TCP ports on a given IP and return open ports."""
\end{verbatim}



\section*2. Meaningful Names

Avoid vague names like x, y, data. Be descriptive.

\begin{verbatim}
def get_user_credentials():
    ...
\end{verbatim}



\section*3. Comments

Explain why something is done, not what is obviously being done.

\begin{verbatim}
# Retry the request if it times out
response = send_request(retry=True)
\end{verbatim}



\section*4. Formatting

Follow PEP 8: 4 spaces per indent, and keep lines under 79 characters.

Use auto-formatting tools like Black or autopep8.



\section*5. Constants

Use ALL_CAPS for values that shouldn't change.

\begin{verbatim}
MAX_ATTEMPTS = 5
\end{verbatim}



\section*6. Avoid Magic Numbers

Give important numbers a name so the code is readable.

\begin{verbatim}
DEFAULT_TIMEOUT = 10
\end{verbatim}



\section*7. List Comprehensions

Use list comprehensions for concise loops.

\begin{verbatim}
ports = [p for p in range(1024) if is_open(p)]
\end{verbatim}



\section*8. Use `with` for File Access

Using `with` ensures the file is closed automatically.

\begin{verbatim}
with open("log.txt") as f:
    data = f.read()
\end{verbatim}



\section*9. Try/Except for Error Handling

Handle errors without crashing your script.

\begin{verbatim}
try:
    connect_to_db()
except ConnectionError:
    print("DB unreachable.")
\end{verbatim}



\section*10. `if __name__ == '__main__'`

Keeps your script modular and reusable.

\begin{verbatim}
def main():
    run_scan()

if __name__ == "__main__":
    main()
\end{verbatim}




\end{document}